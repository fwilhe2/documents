% Copyright 2009-2011,2015 Dominik Wagenfuehr <dominik.wagenfuehr@deesaster.org>
% Dieses Dokument unterliegt der Creative-Commons-Lizenz
% "Namensnennung-Weitergabe unter gleichen Bedingungen 4.0 International"
% [http://creativecommons.org/licenses/by-sa/4.0/deed.de].

% Pakete
\usepackage[utf8]{inputenc}   % UTF8-Kodierung für Umlaute
\usepackage[T1]{fontenc}      % use TeX encoding then Type 1.
\usepackage{ngerman}          % deutsche Silbentrennung
\usepackage{graphicx}         % Einbindung von Grafiken
\usepackage{fancyhdr}         % eigenes Layout einbinden
\usepackage{pdfpages}         % PDF-Seiten einbinden
\usepackage{xifthen}          % Wenn-dann-Abfragen
\usepackage{charter}          % Charter-Schrift
\usepackage{titlesec}         % Anpassung der Überschriften
\usepackage{longtable}        % Tabellen über Seitenumbruch hinweg
\usepackage{setspace}         % Zeilenabstand festlegen
\usepackage{hyperref}         % Hyperlinks und interne PDF-Verweise

% Layout
\hbadness=10000                % unterdrückt unwichtige Fehlermeldungen
\clubpenalty = 10000           % Keine "Schusterjungen"
\widowpenalty = 10000          % Keine "Hurenkinder"
\displaywidowpenalty = 10000

% Linie im Kopf ausblenden und im Fuß eine feine Linie
\renewcommand*{\headrulewidth}{0pt}
\renewcommand*{\footrulewidth}{0.4pt}

% Überschriftenlayout verändern
\titlespacing{\section}{0mm}{2em}{2em}
\titlespacing{\subsection}{0em}{0em}{0em}
\titleformat{\section}{\normalfont\LARGE\scshape}{}{0mm}{\hspace*{\fill}}
\titleformat{\subsection}{\normalfont\Large\bfseries\scshape}{}{0mm}{}[\vskip-1.5em\hrulefill]

% Eigene Fusszeile festlegen, Kopfzeile leeren
\pagestyle{fancy}
\fancyhead{}
\cfoot{\DefVollerName, \DefAbsenderStrasse, \DefAbsenderPLZOrt, \DefTelefon}

% Definition des heutigen Datums
\newcommand*{\HeutigerTag}{\today}

% Definitionen zur eigenen Person
\newcommand*{\DefVollerName}{}
\newcommand*{\DefAbsenderStrasse}{}
\newcommand*{\DefAbsenderPLZOrt}{}
\newcommand*{\DefTelefon}{}
\newcommand*{\DefEMail}{}
\newcommand*{\DefOrtDatum}{}
\newcommand*{\DefGeburtstag}{}
\newcommand*{\DefGeburtsort}{}
\newcommand*{\DefDetails}{}
\newcommand*{\DefAusbildungsgrad}{}
\newcommand*{\DefUnterschriftenDatei}{}
\newcommand*{\DefBewerberFoto}{}

% Definition zum Ueberschreiben der Daten zur eigenen Person
\newcommand*{\VollerName}[1]{\renewcommand*{\DefVollerName}{#1}}
\newcommand*{\AbsenderStrasse}[1]{\renewcommand*{\DefAbsenderStrasse}{#1}}
\newcommand*{\AbsenderPLZOrt}[1]{\renewcommand*{\DefAbsenderPLZOrt}{#1}}
\newcommand*{\Telefon}[1]{\renewcommand*{\DefTelefon}{#1}}
\newcommand*{\EMail}[1]{\renewcommand*{\DefEMail}{#1}}
\newcommand*{\OrtDatum}[1]{\renewcommand*{\DefOrtDatum}{#1}}
\newcommand*{\Geburtstag}[1]{\renewcommand*{\DefGeburtstag}{#1}}
\newcommand*{\Geburtsort}[1]{\renewcommand*{\DefGeburtsort}{#1}}
\newcommand*{\Details}[1]{\renewcommand*{\DefDetails}{#1}}
\newcommand*{\Ausbildungsgrad}[1]{\renewcommand*{\DefAusbildungsgrad}{#1}}
\newcommand*{\UnterschriftenDatei}[1]{\renewcommand*{\DefUnterschriftenDatei}{#1}}
\newcommand*{\BewerberFoto}[1]{\renewcommand*{\DefBewerberFoto}{#1}}

% Absender
\newcommand{\Absender}[1][\normalsize]{%
    \rightline{\textbf{#1\DefVollerName}}
    \rightline{\DefAbsenderStrasse}
    \rightline{\DefAbsenderPLZOrt}
    \rightline{\DefTelefon}
    \rightline{\DefEMail}
}

% Definitionen zum Empfaenger
\newcommand*{\DefFirma}{}
\newcommand*{\DefAbteilung}{}
\newcommand*{\DefAdressatVorname}{}
\newcommand*{\DefAdressatNachname}{}
\newcommand*{\DefAnschriftStrasse}{}
\newcommand*{\DefAnschriftPLZOrt}{}
\newcommand*{\DefAnredeKopf}{}
\newcommand*{\DefAnredeText}{ Damen und Herren}

% Definitionen zum Ueberschreiben der Daten des Empfaengers
\newcommand*{\Firma}[1]{\renewcommand*{\DefFirma}{#1}}
\newcommand*{\Abteilung}[1]{\renewcommand*{\DefAbteilung}{#1}}
\newcommand*{\AdressatVorname}[1]{\renewcommand*{\DefAdressatVorname}{#1}}
\newcommand*{\AdressatNachname}[1]{\renewcommand*{\DefAdressatNachname}{#1}}
\newcommand*{\AnschriftStrasse}[1]{\renewcommand*{\DefAnschriftStrasse}{#1}}
\newcommand*{\AnschriftPLZOrt}[1]{\renewcommand*{\DefAnschriftPLZOrt}{#1}}
\newcommand*{\Anrede}[1]{%
    \ifthenelse{\equal{\DefAdressatNachname}{}}{%
        \renewcommand*{\DefAnredeKopf}{}%
        \renewcommand*{\DefAnredeText}{ Damen und Herren}%
    }{%
        \ifthenelse{\equal{#1}{Herr}}{%
            \renewcommand*{\DefAnredeKopf}{Herrn}%
            \renewcommand*{\DefAnredeText}{r Herr}%
        }{%
            \ifthenelse{\equal{#1}{Frau}}{%
                \renewcommand*{\DefAnredeKopf}{Frau}%
                \renewcommand*{\DefAnredeText}{ Frau}%
            }{}%
        }%
    }%
}

%\errmessage{Unbekannte Anrede! Sollte 'Frau' oder 'Herr' sein}

% Adressat
\newcommand{\Anschrift}{%
    \textbf{\DefFirma}\\
    \ifthenelse{\equal{\DefAnredeKopf}{}}{}{%
        \DefAnredeKopf{}
        \ifthenelse{\equal{\DefAdressatVorname}{}}{}{%
            \DefAdressatVorname{}
        }%
        \ifthenelse{\equal{\DefAdressatNachname}{}}{}{%
            \DefAdressatNachname{} \\
        }%
    }%
    \ifthenelse{\equal{\DefAbteilung}{}}{}{%
        \DefAbteilung\\
    }%
    \DefAnschriftStrasse\\
    \DefAnschriftPLZOrt\\
}

% Makro für Ort und Unterschrift
\newcommand*{\UnterschriftOrt}{%
    \Unterschrift~\\
    \DefOrtDatum\\
}

% Unterschrift als Bilddatei einfügen
\newcommand*{\Unterschrift}{%
    \ifthenelse{\equal{\DefUnterschriftenDatei}{}}{%
        \vspace{2\baselineskip}%
    }{%
        \includegraphics{\DefUnterschriftenDatei}%
    }
}

% Lebenslauf
\newenvironment{Lebenslauf}
{%
    \section{Lebenslauf}
}{%
    \UnterschriftOrt
    \clearpage
}

% Tabelle für Lebenslauf
\newlength{\AbstandAbschnitt}
\newboolean{UnterabschnittBegonnen}
\newenvironment{AbschnittCV}[2][0\baselineskip]
{%
    \subsection{#2}%

    \setlength{\AbstandAbschnitt}{#1}%
    \begin{longtable}{p{0.3\linewidth}p{0.64\linewidth}}%
    \setboolean{UnterabschnittBegonnen}{false}%
}{%
    \end{longtable}%
    \vspace{\AbstandAbschnitt}%
}

% Unterabschnitt in Lebenslauftabelle mit sonstigen Themen
% Achtung, keine Umgebung, sondern ein Kommando!
\newlength{\AbstandUnterabschnitt}
\newcommand{\UnterabschnittCV}[3][0\baselineskip]
{%
    \end{longtable}%
    \ifthenelse{\boolean{UnterabschnittBegonnen}}{%
        \setlength{\AbstandUnterabschnitt}{-3.5\baselineskip}%
    }{%
        \setlength{\AbstandUnterabschnitt}{-1.5\baselineskip}%
    }%
    \addtolength{\AbstandUnterabschnitt}{#1}%
    \vspace{\AbstandUnterabschnitt}%
    \textit{#2}:
    \begin{itemize}
    #3
    \end{itemize}
    \setboolean{UnterabschnittBegonnen}{true}%
    \begin{longtable}{p{0.3\linewidth}p{0.64\linewidth}}%
    \\
}

% Anrede im Brief
\newcommand*{\DefAnrede}{%
    Sehr geehrte\DefAnredeText{}%
    \ifthenelse{\equal{\DefAdressatNachname}{}}{}{
        \DefAdressatNachname{}%
    }%
    ,%
}

% Die Grußformel am Anschreibenende
\newcommand{\Grussworte}{%
Über die Einladung zu einem persönlichen Gespräch freue ich mich und verbleibe

mit freundlichen Grüßen

\Unterschrift~\\[\baselineskip]
}

% Definition der Bewerberstelle
\newcommand*{\DefBewerberstelle}{}

% Definitionen zum Ueberschreiben der Bewerberstelle
\newcommand*{\Bewerberstelle}[1]{\renewcommand*{\DefBewerberstelle}{#1}}

% Betreff
\newcommand*{\Betreff}{\textbf{Bewerbung als \DefBewerberstelle}}

% Etwa Abstand vor dem Ort, um die Seite besser aufteilen zu können.
\newlength{\LenAbstandVorAnschreiben}
\setlength{\LenAbstandVorAnschreiben}{\baselineskip}
\newcommand*{\AbstandVorAnschreiben}[1]{\setlength{\LenAbstandVorAnschreiben}{#1\baselineskip}}

% Abstand zwischen den Adressen im Anschreiben.
\newlength{\LenAbstandZwischenAdressen}
\setlength{\LenAbstandZwischenAdressen}{0\baselineskip}
\newcommand*{\AbstandZwischenAdressen}[1]{\setlength{\LenAbstandZwischenAdressen}{#1\baselineskip}}

% Die Anschreibeseite wird bei Bedarf etwas vergrößert, da es hier keine
% Fußzeile gibt.
\newlength{\LenSeiteVergroessern}
\setlength{\LenSeiteVergroessern}{0\baselineskip}
\newcommand*{\AnschreibenSeiteVergroessern}[1]{\setlength{\LenSeiteVergroessern}{#1\baselineskip}}

\newcommand{\AnschreibenKopf}
{%
    % Anschreiben ohne Fuss- oder Kopfzeile
    \thispagestyle{empty}

    % Seite ggf. etwas vergroessern
    \enlargethispage{\LenSeiteVergroessern}
    
    \begin{spacing}{1.05}
        % Meine Anschrift
        \Absender

        % ggf. will man etwas Abstand zwischen den Adressen.
        \vspace*{\LenAbstandZwischenAdressen}

        % Anschrift der Firma
        \Anschrift

        % ggf. braucht man hier etwas Abstand, je nachdem, wie
        % viel Text man im Anschreiben hat.
        \vspace*{\LenAbstandVorAnschreiben}

        % Ort und Datum
        \rightline{\DefOrtDatum}
    \end{spacing}

}

% Definition des Textes im Anschreiben
\newcommand{\DefAnschreibenAnlage}{}

% Definition zum Setzen des Textes im Anschreiben
\newcommand{\AnschreibenAnlage}[1]{\renewcommand{\DefAnschreibenAnlage}{#1}}

\newenvironment{Anschreiben}
{%
    % Titel und Autor des PDFs werden automatisch festgelegt,
    % was aber erst nach der Definition des Namens geht.
    \hypersetup{%
        pdftitle={Bewerbung bei \DefFirma},
        pdfauthor={\DefVollerName},
        pdfcreator={\DefVollerName}
    }

    \AnschreibenKopf
    \begin{spacing}{1.15}
        \Betreff
        \begin{flushleft}
            \DefAnrede

}{%

            \Grussworte
            
            \vspace{\baselineskip}
            \DefAnschreibenAnlage
        \end{flushleft}
    \end{spacing}
    \clearpage
}

\newcommand{\MeineSeite}
{%
    \section{Zu meiner Person}

    \begin{spacing}{1.3}
        \Absender[\large]
        \vspace{1.5\baselineskip}

        \rightline{geboren am \DefGeburtstag}
        \rightline{in \DefGeburtsort}
        \vspace{0.5\baselineskip}

        \rightline{\DefDetails}

        \rule{10cm}{0.6pt} \\
        {\large \textbf{Ausbildungsgrad}}\\
        \DefAusbildungsgrad\\

        \ifthenelse{\equal{\DefBewerberFoto}{}}{}{%
            \includegraphics[width=5cm]{\DefBewerberFoto}\\
        }

    \end{spacing}
}

\newenvironment{Motivation}
{%
    \section{Über mich und meine Motivation}
 
    \begin{spacing}{1.3}
        \begin{flushleft}

}{%

        \end{flushleft}
    \end{spacing}
    
    \UnterschriftOrt
    \clearpage
}

% Definition der Abschnitte
\newcommand*{\AbschnittAnlage}[1]{\subsection{#1}}

% Definition der Auflistung
\newenvironment{Auflistung}
{\begin{itemize}}
{\end{itemize}}

\newcommand{\punkt}{\item}

% Definition des Anlagenverzeichnisses
\newenvironment{Anlagenverzeichnis}{%
    \section{Anlagenverzeichnis}

    \begin{spacing}{1.1}
}{%
    \end{spacing}
    \clearpage
}

% Anlagen einfuegen
\newcommand*{\AnlageEinfuegen}[1]{\includepdf[pages=-]{#1}}



