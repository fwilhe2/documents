\documentclass{beamer}
\usepackage[ngerman]{babel}
\usepackage[utf8]{inputenc} 

\usetheme{Ilmenau}

%remove navigation symbols
%http://stackoverflow.com/questions/3017030/hiding-the-presentation-controls-in-latex-beamer-presentation
\setbeamertemplate{navigation symbols}{}

\title[]{Systematisches Vorgehen zum Beheben eines Software-Fehlers}
\author{Florian Wilhelm}
\date{7. Januar 2014}

\begin{document}

% Titelseite
\frame{
	\titlepage
}

% Inhaltsverzeichnis
\frame{
	\frametitle{Inhaltsverzeichnis}
	\tableofcontents
}

\section{Problemstellung}

\setcounter{subsection}{1}
\begin{frame} 
  \frametitle{Der Ausbildungsberuf}
    \emph{Fachinformatiker} Fachrichtung \emph{Anwendungsentwicklung}
    
    Dreijährige duale Ausbildung
\end{frame}

\setcounter{subsection}{1}
\begin{frame}
  \frametitle{Der Ausbildungsbetrieb}
   \begin{itemize}
    \item{\emph{Unternehmensberatung Müller Informatik GmbH (UMI GmbH)} mit Sitz in Bruchsal}
    \item{Mittelständisches Systemhaus}
    \item{Bietet volles Spektrum an IT-Dienstleistungen: Von Systemintegration bis Softwareentwicklung}
    \item{Kunden sind vor allem soziale Einrichtungen und Stadtwerke}
    \item{ca. 300 Mitarbeiter}
    \item{Abteilung \emph{Programmentwicklung}}
   \end{itemize}
\end{frame}

\setcounter{subsection}{1}
\begin{frame} 
  \frametitle{Die Ausbildungssituation}
    \emph{Softwarefehler:}
    \leavevmode\hphantom{ } % Newline
    Ein Kunde meldet sich bei unserem Support: \glqq Das Programm arbeitet seit dem Update immer sehr langsam! Dann müssen wir den Computer neu starten!\grqq{}
\end{frame}

\setcounter{subsection}{1}
\begin{frame}
  \frametitle{Der Auszubildende}
   \begin{itemize}
    \item{Max Schmidt, 19 Jahre}
    \item{Seit September 2013 im Unternehmen, 1. Lehrjahr}
    \item{Bildung: Mittlere Reife + Berufskolleg mit FH-Reife}
    \item{Begann schon in der Schule zu programmieren}
    \item{Beherrscht Zehnfingersystem mit 200 Anschlägen pro Minute}
   \end{itemize}
\end{frame}

\section{Lernziele}

\setcounter{subsection}{1}
\begin{frame}
  \frametitle{Richt- und Groblernziele}
   Die Lernziele sind entnommen aus dem \emph{Ausbildungsrahmenplan Fachinformatiker / Fachinformatikerin} der IHK.
   
   \textbf{Richtlernziel:}
   
   \emph{Kundenspezifische Anpassung und Softwarepflege (§ 10 Abs. 2 Nr. 9.1)}
   
   \textbf{Groblernziel:}
   
   \emph{d) Fehler beseitigen}
\end{frame}

\setcounter{subsection}{1}
\begin{frame}
  \frametitle{Feinlernziele}
  \framesubtitle{Kognitiver Bereich}
    Der Auszubildende ist nach der Unterweisung in der Lage \ldots

    \begin{itemize}
      \item{den Umgang mit den im Betrieb verwendeten Softwareentwicklungswerkzeugen}
      \item{eine systematische Vorgehensweise zur Fehlersuche}
      \item{Vorgehensweisen zum logischen Eingrenzen der Fehlerquelle durch Ausschließen von irreführenden Informationen}
    \end{itemize}

    zu kennen.
\end{frame}

\setcounter{subsection}{1}
\begin{frame}
  \frametitle{Feinlernziele}
  \framesubtitle{Affektiver Bereich}
    Der Auszubildende ist nach der Unterweisung in der Lage die \ldots

    \begin{itemize}
      \item{gebotene Sorgfalt im Umgang mit Kundeninformationen}
      \item{Einhaltung von Datenschutzbestimmungen}
      \item{Einhaltung des Urheberrechts am Quellcode der Software}
    \end{itemize}

    zu beachten.
\end{frame}

\setcounter{subsection}{1}
\begin{frame}
  \frametitle{Feinlernziele}
  \framesubtitle{Psychomotorischer Bereich}
   Wird nicht angesprochen in dieser Unterweisung.
   
   (Sicherer Umgang mit Maus und Tastatur (Zehnfingersystem) wird vorausgesetzt.)
\end{frame}

\section{Die gewählte (Unterweisungs-)Methode}

\setcounter{subsection}{1}
\begin{frame}
  \frametitle{Lehrgespräch}
  
    Es wird ein handlungsorientiertes Lehrgespräch durchgeführt.
  
    Ablauf:
  
    \begin{enumerate}
      \item{Informationsinput durch Ausbilder}
      \item{Einbeziehung von Erfahrungen des Azubi}
      \item{Sammeln der Fakten}
      \item{Zusammenfassen und gegebenenfalls Ergänzen durch weitere Informationen}  
    \end{enumerate}
    
    Modifikation: Eigenarbeitsphase des Azubi zwischen Punkt 3 und 4
\end{frame}

\setcounter{subsection}{1}
\begin{frame}
  \frametitle{Alternative Methoden}
    \framesubtitle{Vier-Stufen-Methode}
    
    \begin{enumerate}
      \item{Vorbereiten}
      \item{Vormachen}
      \item{Nachmachen}
      \item{Üben}  
    \end{enumerate}    
      
      Nachteil: Zu sehr fixiert auf Nachmachen einer Tätigkeit, schult nicht eigenständiges Denken
\end{frame}

\setcounter{subsection}{1}
\begin{frame}
  \frametitle{Alternative Methoden}
    \framesubtitle{Projektmethode}
   
    \begin{enumerate}
      \item{Themenfindung}
      \item{Planung des Projektablaufs}
      \item{Durchführung}
      \item{Kontrolle der Ergebnisse}  
      \item{Dokumentation}
    \end{enumerate}        
    
      Nachteil: Passt weniger gut zum natürlichen Arbeitsablauf; eher für Entwicklung neuer Programmteile geeignet als zur Fehlersuche
\end{frame}

\section{Verlaufsplan}

\setcounter{subsection}{1}
\begin{frame}
  \frametitle{Vorbereitung}
  
        \begin{tabular}{|p{3.3cm}|p{3.3cm}|p{3.3cm}|}
        \hline
        WAS & WIE & WARUM \\ \hline
        Hotlineanruf \mbox{entgegennehmen} & Telefonisch & Kunde hat ein \mbox{Problem} \\ \hline
       \end{tabular}  
            
\end{frame}

\setcounter{subsection}{1}
\begin{frame}
  \frametitle{Teil 1: Input vom Ausbilder}
  
        \begin{tabular}{|p{3.3cm}|p{3.3cm}|p{3.3cm}|}
        \hline
        WAS & WIE & WARUM \\ \hline
        Azubi herbeirufen und begrüßen & Verbal & Spannende Aufgabe für ihn \\ \hline
        Fehler demonstrieren & Verbal / am PC & Problembewusstsein wecken \\ \hline
       \end{tabular}

\end{frame}

\setcounter{subsection}{1}
\begin{frame}
  \frametitle{Teil 2: Einbeziehen der Erfahrungen des Azubi}

        \begin{tabular}{|p{3.3cm}|p{1.3cm}|p{5.3cm}|}
        \hline
        WAS & WIE & WARUM \\ \hline
        Nach Vorkentnissen/Erfahrungen fragen & Verbal & Motivieren; in Problemanalyse einbinden \\ \hline
        Nach Vermutungen für Fehlerursache fragen & Verbal & Zum Mitdenken motivieren \\ \hline
       \end{tabular}        


\end{frame}

\setcounter{subsection}{1}
\begin{frame}
  \frametitle{Teil 3: Sammeln der Fakten}
  
        \begin{tabular}{|p{3.3cm}|p{3.3cm}|p{3.3cm}|}
        \hline
        WAS & WIE & WARUM \\ \hline
        Diskussion über mögliche Fehlerquellen & Whiteboard zur \mbox{Visualisierung} & Schulen des analytischen Denkens und logischen Schließens \\ \hline
        Diskussion einer Lösungsmöglichkeit & Whiteboard zur \mbox{Visualisierung} & Vorbereitung für vom Azubi zu erbringende Leistung \\ \hline
       \end{tabular}

  

\end{frame}

\setcounter{subsection}{1}
\begin{frame}
  \frametitle{Einschub: Eigenarbeit des Azubis}
  
        \begin{tabular}{|p{3.3cm}|p{1.3cm}|p{5.3cm}|}
        \hline
        WAS & WIE & WARUM \\ \hline
        Azubi ca. 4 Stunden Zeit lassen um das Problem zu lösen & PC & Größter Lerneffekt ist zu erwarten wenn der Azubi das Problem selbst lösen kann \\ \hline
       \end{tabular}  

\end{frame}

\setcounter{subsection}{1}
\begin{frame}
  \frametitle{Teil 4: Zusammenfassen und Ergänzen}
  
        \begin{tabular}{|p{3.3cm}|p{3.3cm}|p{3.3cm}|}
        \hline
        WAS & WIE & WARUM \\ \hline
        Besprechen der \mbox{Arbeitsergebnisse} & Verbal / PC & Qualitätskontrolle; Ggf. um Hilfestellung leisten zu können \\ \hline
        Azubi Software-Tests durchführen lassen & PC & Qualitätskontrolle; Lernziel \glqq Sorgfältiges \mbox{Arbeiten}\grqq{} \\ \hline
        Fakten von Azubi auf Whiteboard zusammenfassen und in Berichtsheft übernehmen lassen & Verbal / Whiteboard zur Visualisierung & Lernzielkontrolle: Wenn Azubi es verstanden hat kann er es erklären \\ \hline         
       \end{tabular}

\end{frame}

\section{}
\begin{frame}
  Vielen Dank für Ihre Aufmerksamkeit
\end{frame}

\end{document}
